\section{Research Challenges and Goals}
\label{sec:goal}

\subsection{Feasibility}
We show the feasibility of this project by the following three aspects:
\begin{itemize}
    \item Data: Open-source software and version control systems provide sufficient meta-data for analysis.
    \item Tool: Existing tools, such PMD, SonarQube, FindBugs and CAST provide various quality metrics.
    \item Techniques: The machine learning and natural language methods required in this research already exist.
\end{itemize}
With all above provided, we believe this plan will succeed in 3~5 years.
The midterm milestone is a reasonable high prediction accuracy from the machine learning model.
The final milestone is the completion of the new systematic coding standard and development guidelines.


\subsection{Stage One}
Previous researchers didn't study the correlation between the change type and the code as well as their impact on the software quality. 
In this stage, we start with establishing categorization of commit changes in open source software repositories.
We evaluate the quality of those changes by obtaining quality metrics from static analysis tools.
To assess the correlation between the quality and the categories, we plan to train a machine learning model, in addition to standard mathematical correlation analyses.

\subsection{Stage Two}
In this stage, although we have categorized the commits changes, we need further insight on the distinguishing different categories.
This is because it has been shown that high-level categories have overlaps with each other.
In this stage, we will be working on removing the ambiguity of the categories by analyzing the code changes within the commits rather than the commit messages and manual categorizing.
Once this is done. We will investigate the correlation between the categories and changes in code to reveal whether they have close connection and how those changes impact software quality.

\subsection{Stage Three}
In the final stage, with the refined categories, we will start construct guidelines for developers on how they can better contribute to open source software when they make different type of changes.
In addition, we will create an index to indicate how different code patterns impact the software quality.
We will conclude this research by completing and validating these two aspects. 

