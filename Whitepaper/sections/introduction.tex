\section{Introduction}
\label{sec:introduction}

Before explaining any details of this agenda, we introduce the context of this research, including open source software, version control systems and different aspects of software repository mining that should be considered. 
These aspects include commit impacts, purposes, commit messages, code patterns and software quality.
We conclude this section with an outline of the rest of this white paper. 

%%%%%%%%%%%%%%%%%%%%%%%%%%%%%%%%%%%%%%%%%%%%
\subsection{Open Source Software and Version Control Systems}
Using open-source repositories has long been a common way to develop software.
Some of these projects are on an industrial scale.
% on an industrial scale
As the scale has grown far beyond the level that an individual can control and manage, how to efficiently conduct quality control and project management is critical.

Most industrial-scale software is developed by iterative contributions from project teams, through ICSM \cite{icsm2014}, Agile \cite{agile2004}, DevOps \cite{devops2016} or other process models. 
In the iterations, version control systems, such as Git and SVN, play a critical role by enabling and facilitating the concurrent contributions from developers.
Each revision, or commitment (hereafter ``commit''), contains diffs which are the lines developers change.

%hereafter

These changes can be made by developers from different areas of the world, at different times, have different purposes and have different impacts on the software \cite{qrs2020}, be they negative or positive. 
Thus, it is necessary to investigate how these differences influence software quality, and thus to better control the quality during the development and maintenance phases. 

Focusing on the different purposes of commits, this white paper investigates how different types of commits impact software quality and propose guidelines for improving. 


%%%%%%%%%%%%%%%%%%%%%%%%%%%%%%%%%%%%%%%%%%%%
\subsection{Level of Commit Impacts}
In projects, some commits impact software quality more than others.
For example, commits that change core modules, which modify system functionalities are more impactful than those that contain only a few lines of documentation fixes.

The level of impact can be defined in various ways to specify what to investigate. For example, in previous studies, researchers have defined impactful commits by whether they are in the core module \cite{pooyan_esem, pooyan_qrs}.
We believe that the more critical the commits are, the earlier they should be taken care of, in the sense of quality control and management. 


%%%%%%%%%%%%%%%%%%%%%%%%%%%%%%%%%%%%%%%%%%%%
\subsection{Purposes of Commits}
While the levels of impact differ, the commits also vary in their purposes.
For example, some commits merely add a few lines of documentation or comments to code while others refactor the entire code structure or made module-level modifications.


\comment{A better transition here.}
It is common for developers to upload single-purpose commits\footnote{https://www.freshconsulting.com/atomic-commits/}. 
However, in commits where developers refactor code, add new dependencies, or apply minor fixes, the commits tend to grow beyond their intended task.
In this case, commits become multi-purpose, and it has been proven in a previous study \cite{qrs2020} that multi-purpose commits have negative impacts on software quality, compared to single purposes ones.
%Each commit includes a prescriptive message documenting the changes made\footnote{https://git-scm.com/book/en/v2/Distributed-Git-Contributing-to-a-Project}, in practice with varying degrees of efficacy. 
In addition, it has been shown that some types of commits, such as ``feature add'', are more likely to have negative impacts on software quality.
Thus, it is important to investigate how different types of commits impact various aspects of quality and how they are related to the other metadata of software to create new guidelines for developers, thus help them improve the quality.

To achieve this, we review previous works that categorize commits and find that some of them have produced taxonomies for commit purposes \cite{Hindle_cate,alali_2008,Dragan,Swanson, Mauczka, Hindle_auto,qrs2020} which we will evaluate and adopt.

%Thus, we start with the commit purpose taxonomy from Hindle et al. \cite{Hindle_cate}, refine, standardize, and expand it to make it applicable for small commits. 
%However, other studies \cite{x} report that 75\% of software development budgets are dedicated to maintenance.


%%%%%%%%%%%%%%%%%%%%%%%%%%%%%%%%%%%%%%%%%%%%
\subsection{Commit Message}
One critical piece of the project metadata is the commit message. 
When developers push changes to online repositories, they are required to add commit messages to explain their changes.

These messages provide important clues for understanding the purposes of those commits. 
As a result, it is important to analyze commit messages to help understand the purposes of the commits and categorize them.

%%%%%%%%%%%%%%%%%%%%%%%%%%%%%%%%%%%%%%%%%%%%
\subsection{Code Pattern}
The commit messages are, in effect, summaries from developers about their code changes.
%in effect


%The code can be written in different manners, and which one is the best to
%A good example is Dr. Liscov and her contribution to code hierachy.

Also, there are works aiming at how to generate code automatically which lead to inventions of code generation tools.
Both hand-typed and auto-generated code are related to this research.
For hand-written, it is possible to use them to train a prediction model by using NLP techniques to automatically extract keywords from code changes, which explain their purposes.
On the other hand, the outcome of this research, for example the guidelines, are useful when developing code generation tools.
\comment{Think about whether to add this application into abstract.}


%%%%%%%%%%%%%%%%%%%%%%%%%%%%%%%%%%%%%%%%%%%%
\subsection{Software Quality}
The final goal of the research is to improve software quality.
Thus, defining assessment guidelines for software quality is one of the most important issues of this research.

Software quality is evaluated using different tools depending on one's purpose. 
For example, COCOMO and COCOMO II \cite{cocomo1995} evaluate software with respect to their cost. 
PMD \footnote{https://pmd.github.io/}, SonarQube \footnote{https://www.sonarqube.org/} and FindBugs \footnote{http://findbugs.sourceforge.net/} define metrics based on software metadata and algorithms to evaluate security, vulnerability and bugs. 
CAST software \footnote{https://www.castsoftware.com} provides architecture evaluations in addition to other metrics.

In this research, we analyze software quality using tool-based metrics and compilability and will create a new metric showing the category distribution of commit purposes.
%we will use tool-based software metrics and compilability when we evaluate software quality.
%Furthermore, when we finalize this research, the categories and category-distributions of commits will also be metrics to evaluate software quality. 

%%%%%%%%%%%%%%%%%%%%%%%%%%%%%%%%%%%%%%%%%%%%
\subsection{Outline of This Research White Paper}
In this white paper, we evaluate previous work on related areas of this research, such as commit messages, code patterns and software quality representation, as well as propose a purpose-oriented categorizational analysis approach to analyze commits and software quality.

Our goal is to construct a reasonable categorization for commits based on their purposes, and investigate their relations to software quality and consequently find a way to improve them.

The sections of this white paper are organized as follows:
\begin{itemize}
    \item Section \ref{sec:data} introduces our current data set and additional data we plan to collect for current projects and from new projects.
    %\item Section \ref{sec:data} introduces our current data set and what we want to collect beyond it to support further analysis.
    %specify "what" here
    \item Section \ref{sec:type} discusses previous works on categorizing commits, either by their purpose, size or other criteria and how we can apply them to analyzing how different types of commits impact software quality. Furthermore, the section also discusses the critical issues in the process of categorization that need extra effort to deal with.
    \item Section \ref{sec:message} discusses previous works in analyzing and extracting information from commit messages, auto-generating them, as well as using commit messages to help categorize commits and analyzing software quality.
    \item Section \ref{sec:pattern} illustrates how we can use code changes to help categorize commits.
    \item Section \ref{sec:quality} explains our way of assessing software quality, including tool-based analysis and additional metrics.
    \item Section \ref{sec:goal} presents the plan and final goals of this research and how this research will serve to improve software quality across the field.
    \item Section \ref{sec:conclusion} concludes the white paper.
    \item Section \ref{sec:threats} lists potential threats to the validity of this research plan.
\end{itemize}


