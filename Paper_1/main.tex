% !TeX root = ./main.tex
\documentclass[10pt,conference]{IEEEtran}
\IEEEoverridecommandlockouts
% The preceding line is only needed to identify funding in the first footnote. If that is unneeded, please comment it out.
\usepackage{cite}
\usepackage{amsmath,amssymb,amsfonts}
\usepackage{algorithmic}
\usepackage{graphicx}
\DeclareGraphicsExtensions{.pdf,.png,.eps}
\usepackage{textcomp}
\usepackage{xcolor}
\usepackage[T1]{fontenc}

\usepackage{xspace}
%new packages from Pooyan's dissertation start
\usepackage{multicol,lipsum}
\usepackage{multirow}
%new packages from Pooyan's dissertation end
\def\BibTeX{{\rm B\kern-.05em{\sc i\kern-.025em b}\kern-.08em
    T\kern-.1667em\lower.7ex\hbox{E}\kern-.125emX}}
    
\definecolor{update}{RGB}{0, 127, 0}
\definecolor{delete}{RGB}{200, 0, 0}
\definecolor{Michael}{RGB}{70, 145, 235}
\definecolor{request}{RGB}{255, 165, 0}
\definecolor{sitao}{RGB}{0,130,0}
\newcommand{\pooyan}[1]{{\leavevmode\color{blue}[Pooyan: #1]}\xspace}
\newcommand{\delete}[1]{{\leavevmode\color{delete}[Deleted: #1]}\xspace}
\newcommand{\michael}[1]{{\leavevmode\color{Michael}[Michael: #1]}\xspace}
\newcommand{\sitao}[1]{{\leavevmode\color{sitao}[sitao: #1]}\xspace}
\newcommand{\update}[1]{{\leavevmode\color{update}[Updated: #1]}\xspace}
\newcommand{\request}[1]{{\leavevmode\color{request}[Requested: #1]}\xspace}
\newcommand{\review}[1]{{\leavevmode\color{orange}[Review: #1]}\xspace}
\newcommand{\RQsize}{\emph{What are the differences between small and large commits in terms of their purpose?}\xspace}
\newcommand{\RQtype}{\emph{Is there any correlation between change type and software compilability?}\xspace}
\newcommand{\RQmetrics}{\emph{Do quality metrics change differently when the software is uncompilable?}\xspace}
%new commands from Pooyan's dissertation start
\newcommand{\algsmall}{\algsetup{linenosize=\small}\small}


\newcommand{\mto}{\mathit{mto}\xspace}
\newcommand{\clusters}[1]{{C_{#1}}}
\newcommand{\entities}[1]{{E_{#1}}}

\newcommand{\abs}[1]{{abs(#1)}}
\newcommand{\card}[1]{{\left\vert #1\right\vert}}

\newcommand{\TODO}[1]{{\leavevmode\color{blue}[: #1]}\xspace}

\newcommand{\RQCompilationEfficiency}{\emph{How effective is my approach in reaching the maximum compilation?}\xspace}
%		 and identifying the impact of commits on compilability?}\xspace}
\newcommand{\RQBrokenSequence}{\emph{What are the characteristics of sequences of uncompilable commits?}\xspace}
\newcommand{\RQBrokenPrediction}{\emph{Is it feasible to predict uncompilability based on commit metadata?}\xspace}
\newcommand{\RQBrokenWhyHow}{\emph{Why do developers commit broken code and how to prevent it?}\xspace}

\newcommand{\RQChangeNeutrals}{\emph{How do quality metrics change when the software is compilable?}\xspace}
\newcommand{\RQQualityProbability}{\emph{How important is analyzing every commit from multiple perspectives?}\xspace}
\newcommand{\RQQualityBroken}{\emph{How do quality metrics change when the software is uncompilable?}\xspace}
\newcommand{\RQQualityCompilability}{\emph{Is there any difference between commits in terms changing quality metrics based on their impact on compilability?}\xspace}

\newcommand{\RQQualityEffectiveness}{\emph{How effective is my approach in identifying change in quality metrics?}\xspace}
\newcommand{\RQArchitectureEvolutionCommit}{\emph{How frequently does the architecture of software change over its commit-history?}\xspace}

\newcommand{\ChapterTitleCompilability}{Compilability Over Commit History\xspace}
\newcommand{\ChapterTitleQuality}{Compilability and Its Impact on Software Quality\xspace}
\newcommand{\ChapterTitleDiscussions}{Discussions\xspace}

\newcommand{\impactful}{\emph{impactful}\xspace}
\newcommand{\iparent}{\emph{impact-parent}\xspace}
\newcommand{\ichild}{\emph{impact-child}\xspace}

\newcommand{\orphan}{\emph{orphan}\xspace}
\newcommand{\simple}{\emph{simple}\xspace}
\newcommand{\merge}{\emph{merge}\xspace}

\newcommand{\broken}{\emph{broken}\xspace}
\newcommand{\solid}{\emph{solid}\xspace}
\newcommand{\breaker}{\emph{breaker}\xspace}
\newcommand{\carrier}{\emph{carrier}\xspace}
\newcommand{\fixer}{\emph{fixer}\xspace}
\newcommand{\neutral}{\emph{neutral}\xspace}

\newcommand{\length}{\emph{length}\xspace}
\newcommand{\duration}{\emph{duration}\xspace}
\newcommand{\involvement}{\emph{involvement}\xspace}

\newcommand{\external}{\emph{external}\xspace}
\newcommand{\affiliated}{\emph{affiliated}\xspace}

\newcommand{\target}{\emph{target}\xspace}
%new commands from Pooyan's dissertation ends

\begin{document}

\title{To be Defined}

\author{
\IEEEauthorblockN{Anonymous}
\IEEEauthorblockA{\textit{Department of Computer Science} \\
\textit{University of Southern California}\\
%City, Country \\
Anonymous@usc.edu}
\and
\IEEEauthorblockN{Anonymous}
\IEEEauthorblockA{\textit{Department of Computer Science} \\
\textit{University of Southern California}\\
%City, Country \\
Anonymous@usc.edu}
}


\maketitle

\begin{abstract}
Nowadays, it is common to see software being developed with support from online version control systems (VCS), such as Git. This makes it easier to track the development as well as to evaluate the quality of the software.
At the same time, open source makes the projects more vulnerable \cite{} because the developers can be from different organizations with different level of skills.
Thus, open source software (OSS) quality investigation, control and improvements are one of the critical issues in the area of software engineering, and repository mining is one of the approaches to conduct them.
Thus far, by repository mining, various researches have been conducted to assess the quality of software, but there is still space for investigations about analyzing how developer behaviors impact software quality and to extend the concept of software quality.
In this research area, little work has been done in classification-based study on commits, focusing on types, commit message and code patterns, to analyze their relations with and impact on software quality.
In this white paper, we list fields where and how further research can be done for the sake of a better software quality. 

\end{abstract}

\begin{IEEEkeywords}
Software Engineering, Software Maintenance, Software Quality, Open Source Software
\end{IEEEkeywords}

\section{Introduction}
Most industry-scale software are developed under iterative contributions from the project teams \cite{}, through ICSM, Agile, DevOps or other process models. 
In the iterations, version control systems, such as Git and SVN, play a critical role by enabling and facilitate the concurrent contributions from developers.
Each revision, or commitment (referred as ``commit'' in the rest of the paper) contains diffs which are the lines developers change.
These changes 

%%%%%%%%%%%%
Industry-scale software evolves through thousands of iterative code changes. Version control systems, such as Git and SVN, allow these changes to take place in parallel, with multiple developers contributing to a shared repository. Each contribution is captured in a commit -- a collection of file diffs that indicate modified code.
%One aspect of software evolution is commit purpose \cite{Hindle}.
Developers author commits for different purposes.
For example, a commit can edit the build configuration, fix a bug, or refactor existing code.
It is common practice for developers to author atomic commits \footnote{https://www.freshconsulting.com/atomic-commits/} that have a singular purpose. However, refactoring, added dependencies, minor fixes, and scope creep can cause the commit to grow beyond its intended task. Each commit includes a prescriptive message documenting the changes made \footnote{https://git-scm.com/book/en/v2/Distributed-Git-Contributing-to-a-Project}, in practice with varying degrees of efficacy. 

A software revision created by a commit is expected to be compilable. However, uncompilability can occur due to careless development -- failure to compile the software locally prior to pushing to the shared repository. It can also result from variations in build environments, incompatibility across overlapping changes made by multiple developers, or changes in upstream dependencies. 
The presence of compile errors inhibits bytecode and dynamic software analysis, as well as static analysis when the code is unparsable \cite{Behnamghader_ESEM}.
Previous studies \cite{Behnamghader_ESEM, 8170083, Hassan2017ESEM, SMR:SMR1838} have shown that even in popular open-source projects maintained by major software organizations, build-breaking commits can occur.

Behnamghader et al.\cite{Behnamghader_QRS} explore the qualitative properties of uncompilable commits.
Further insight into the types of commits that most frequently cause build errors can help to inform better development practices. Additionally, this correlation can be used as a part of future methods for predicting and preventing uncompilable commits. 
Analyzing the degradation of software quality over multiple uncompilable commits highlights the long-term negative impact of careless development, and the importance of fixing build errors immediately after they take place.
Also, understanding the purposes of those uncompilable commits can result in preparing guidelines to avoid such degradation in the future.

Previous research has produced taxonomies for commit categorization \cite{Hindle_cate,Alali,Dragan,Swanson, Mauczka, Hindle_auto}. While sharing some common category terminologies and definitions, these taxonomies cannot be directly adopted for understanding the purposes of commits.
The taxonomies that integrate commit purpose have only been adapted to large commits \cite{Hindle_cate,Hindle_auto} which lack the specificity to classify smaller code changes \cite{Swanson}.
%Thus, we start with the commit purpose taxonomy from Hindle et al. \cite{Hindle_cate}, refine, standardize, and expand it to make it applicable for small commits. 
%However, other studies \cite{x} report that 75\% of software development budgets are dedicated to maintenance.

In this paper, we analyze the differences in software quality metrics before and after sequences of uncompilable commits and compare them with the compilable sequences to study whether software quality changes differently over uncompilable periods. To gain a deeper insight into this, we explore the relation between commit size, purpose and uncompilability.
We employ the commit purpose taxonomy from Hindle et al. \cite{Hindle_cate}, refine, standardize, and expand it to make it applicable for small commits. 
Our goal is to further understanding of the causes and effects of uncompilability in open-source systems, a significant problem that resists explanation by common software quality analysis tools. This work consists of the following contributions: 
\begin{itemize}
    \item A methodology for assessing software quality over sequences of uncompilable commits.
    \item Findings on the evolution of software quality over sequences of uncompilable commits.
    \item A definition for ``independent changes'' to help categorize commits.
    \item A taxonomy for categorizing commit purpose that can be used across commits of varying size.
    \item A dataset of 914 commits tagged using this taxonomy.
    \item Findings on the relationship between commit size and purpose.
    \item Findings on the incidence of commit purpose over uncompilable commits.  
\end{itemize}

The remainder of the paper is organized as follows.
Section \ref{sec:related-work} summarizes the related works that analyze compilability over commit history and the related works that categorize commits based on commit purpose. 
Section \ref{sec:research-questions} discusses our research questions.
Section \ref{sec:data-collection} discusses our data collection process.
Section \ref{sec:approach} discusses our approach with regard to refining an existing taxonomy, to dealing with inconsistencies and ambiguities in the tagging process, and to studying software quality evolution over uncompilable periods. 
Section \ref{sec:results} discusses the results of our analysis with respect to the research questions.
Section \ref{sec:threats-to-validity} focuses on the threats to validity of our analysis and Section \ref{sec:conclusions} concludes the paper.





\section{Related Works}
\cite{Hindle_auto}

%\input{sections/dataset.tex}

\section{Research Questions}

\subsection{Do different type of changes have different level of impact on software quality?}

\subsection{Do longer commit messages imply higher software quality?}

\subsection{Is there any relation between commit message lengths and commit types?}

Are those relations causal relations or correlations?

\input{sections/dataset.tex}

 \section{Approach}
 To be defined


\section{Results}
To be defined







\section{Threats to Validity}
TBD

\section{Conclusions}
To be defined

\section*{Acknowledgment}
This material is based upon work supported in part by the U.S. Department of Defense through the Systems Engineering Research Center (SERC) under Contract No. HQ0034-13-D-0004 Research Task WRT 1016 --- ``Reducing Total Ownership Cost (TOC) and Schedule.'' SERC is a federally funded University Affiliated Research Center managed by Stevens Institute of Technology. 
%\section*{Acknowledgment}
%This material is based upon work supported in part by the X through the Y under Contract Z.
%X is a Y managed by Z.


\medskip
\bibliographystyle{./bibliography/IEEEtran}
\bibliography{./bibliography/IEEEabrv,./bibliography/IEEEexample,./bibliography/references}

\vspace{12pt}
\color{red}
%IEEE conference templates contain guidance text for composing and formatting conference papers. Please ensure that all template text is removed from your conference paper prior to submission to the conference. Failure to remove the template text from your paper may result in your paper not being published.


\end{document}