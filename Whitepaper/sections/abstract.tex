Developing software with the source code open to the public is very common; however, similar to its closed counterpart, open-source has quality problems, which cause functional failures, such as program breakdowns, and non-functional, such as long response times.
Previous researchers have revealed when, where, how and what developers contribute to projects and how these aspects impact software quality. 
However, there has been little work on how different categories of commits impact software quality.
To improve open-source software, we propose this research agenda to investigate how its quality is impacted by commits of different purposes.
By identifying these impacts, we will establish a new set of guidelines for committing changes, thus improving the quality.
