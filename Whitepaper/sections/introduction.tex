\section{Introduction}
\label{sec:introduction}


%%%%%%%%%%%%%%%%%%%%%%%%%%%%%%%%%%%%%%%%%%%%
\subsection{Open Source Software and Version Control System}
It has been a long time since developing software in an open source repository became a common way of developing software. 
In those project, some of them are of industrial scale.
As the scale has grown far beyond the level that an individual can control and manage, how to efficiently conduct quality control and project management is one of the critical issue in open source software development.

Most industry-scale software are developed by iterative contributions from the project teams \cite{}, through ICSM, Agile, DevOps or other process models. 
In the iterations, version control systems, such as Git and SVN, play a critical role by enabling and facilitate the concurrent contributions from developers.
Each revision, or commitment (referred as ``commit'' in the rest of the paper) contains diffs which are the lines developers change.

These changes can be made by developers from different area of the world, at different times, with different purposes and have different level of impacts on the software \cite{qrs2020}, correspondingly having negative or positive impact on the software quality. 
Thus, it is necessary to investigate how these differences influence the software quality to understand it and control it better during the development and maintenance phases. 

Focusing on the different purposes of commits, this research white paper investigate how different types of commits, with respect to their purposes, impact the software quality and propose what we can do to control it. 


%%%%%%%%%%%%%%%%%%%%%%%%%%%%%%%%%%%%%%%%%%%%
\subsection{Level of Commit Impacts}
In all commits to a project, some commits can be less while others being more impactful on the project and software quality.
Some projects have multiple modules, one of which is core modules while others are less critical to the entire project. 
Moreover, some commits many contains only a few documentation fixes while others do hundreds of lines of modifications.
These commits in this situation can have different level of impact on the entire project.

The level of impact can be defined in various ways to specify what to investigate. For example, in some previous study, researchers defined impactful commits by whether they are in the core module \cite{pooyan_esem, pooyan_qrs}.
We believe that the more critical are the commits, the earlier they should be taken care of, in the sense of quality control and management. 


%%%%%%%%%%%%%%%%%%%%%%%%%%%%%%%%%%%%%%%%%%%%
\subsection{Purposes of Commits}
While the level of impact can be different, the type of commits can also be different.
For example, some commits add a few lines of documentation or comments to code while others can refactoring the code structure or made module-level modifications.

It is common for developers to upload single-purpose commits\footnote{https://www.freshconsulting.com/atomic-commits/}. 
However, in commits where changes such as refactoring, adding new dependencies, minor fixing happen, the commits tend to grow beyond its intended task. 
Each commit includes a prescriptive message documenting the changes made\footnote{https://git-scm.com/book/en/v2/Distributed-Git-Contributing-to-a-Project}, in practice with varying degrees of efficacy. 

In addition, it has been shown that different types, by purpose, of commits have impact on one aspect of the software quality, which is compilability\cite{qrs2020}.
Thus, it is worth working on to investigate how different types of commits impact other aspects of software quality and how they are related to the other metadata of software to acquire a new series guidelines by which developers can following to help improve the overall software quality.

Previous research has produced and refined taxonomies for commit categorization \cite{Hindle_cate,Alali,Dragan,Swanson, Mauczka, Hindle_auto,qrs2020} which is an option for this research for understanding the purposes of commits.

%Thus, we start with the commit purpose taxonomy from Hindle et al. \cite{Hindle_cate}, refine, standardize, and expand it to make it applicable for small commits. 
%However, other studies \cite{x} report that 75\% of software development budgets are dedicated to maintenance.


%%%%%%%%%%%%%%%%%%%%%%%%%%%%%%%%%%%%%%%%%%%%
\subsection{Commit Message}
One critical piece of the project metadata is the commit message. 
When developers push some change to the online repository, a commit message is usually required to explain what they have change in the commit they want to push.

These messages provide important clues for understanding the purposes, thus the types of those commits. 
As a result, in order to conduct this categorizational study, a further insight is needed in analyzing commit message, about how they are written by human, how they are generated by some agents and how they are related to the software quality, for example, readability and maintenability.
\comment{Some more references to be added here.}


%%%%%%%%%%%%%%%%%%%%%%%%%%%%%%%%%%%%%%%%%%%%
\subsection{Code Pattern}
The commit messages are summaries from developers about their work while the code is more fundamental and significant part of the commits. 
How code should be written, for the sake of better software quality, has long been a hard problem to attack.
\comment{Add some more references here.}
A good example of this kind of work can be Dr. Liscov and her contribution to code hierachy.

Also, there are some work aiming at how to generate code automatically, which can be seen in testing code generation tools \comment{Some more contents to be added here.}

Both hand-typed code and auto-generated code are related to this research.
For hand-written code, it is possible to use them to train a prediction model by using NLP techniques to automatically extract keywords from code changes, thus supporting the categorization.
On the other hand, auto-generation can be one of the applicational practice of the research results, since one of the goal of the research is to build guidelines of how to better contribute to software with respect to different types of changes.


%%%%%%%%%%%%%%%%%%%%%%%%%%%%%%%%%%%%%%%%%%%%
\subsection{Software Quality}
The final goal of the research is to improve software quality in the sense of how to contribute to open source software. 
Thus, defining the way to assess software quality of projects is one of the most important issues of this research.

As we know, software quality can be evaluated from different directions. 
For example, COCOMO and COCOMO II \comment{Add some more contents and references here} evaluate software with respect to their cost.
Some tools, such PMD, SonarQube, FindBugs and CAST evaluate software by defining metrics based on software metadata and some algorithms, reflecting the level of security, vulnerability and architecture. \comment{Some more references here, including the official site links}.

In this research, the tool-based software quality metrics is major representation of quality we plan to use.
In a more general sense, we use compilability \comment{Add something about robustness, maintenability, feasibility}, and on the completion of this research, plan to use categories and category distributions of commits as a novel set of software quality metrics.


%%%%%%%%%%%%%%%%%%%%%%%%%%%%%%%%%%%%%%%%%%%%
\subsection{Outline of This Research White Paper}
In this white paper, we evaluate previous work on close research areas of this research: commit message, code parttern and software quality representation and propose a purpose-oriented categorizational analysis approach on commits and software quality.

Our goal is to construct a reasonable categorization for commits, based on their purposes, investigate its relation with software quality and consequently find a way to improve the software quality.

The sections of this white paper is organized as following:
\begin{itemize}
    \item Section \ref{sec:data} introduce our current data set, which we will use for the research, following by a plan of what we want to collect in addition to current set to support further analysis.
    \item Section \ref{sec:type} discusses previous works in categorizing commits, either by their purposes, sizes and other criteria and how we can apply them to analyze impact on software quality. Furthermore, it also discusses the critical issues in the process of categorization that need extra effort to deal with.
    \item Section \ref{sec:message} discusses previous works in analyzing, extracting information, and auto-generation of commit messages \comment{some more sub-directions can be added here} and how we can use commit message to help categorization and analyzing quality metrics.
    \item Section \ref{sec:pattern} discusses previous works in code patterns in the context of natural language processing and how we can use them to support the categorization and how they are related to software quality.
    \item Section \ref{sec:quality} discusses our way of assessing software quality, including tool-based analysis and extra new metrics.
    \item Section \ref{sec:goal} discusses the plan and final goals of this research and how it can be applied practically to serve software development for a better quality.
    \item Section \ref{sec:conclusion} concludes the white paper.
\end{itemize}


