\section{Research Plan and Goals}
\label{sec:goal}

In previous sections, we present the data set and tools we will use in this research, investigate the sub-areas related to this research topic, and identify the corresponding research challenges.
To resolve the challenges, we come up with the following research plan and explain the benefits of this research:

\subsection{Stage One}
Few researchers have studied the correlation between the change type and the code, or how change type impacts quality. 
In this stage, we start by refining the existing categorization of commit changes in open source software repositories.
We evaluate the quality of those changes by obtaining quality metrics from static analysis tools, to demonstrate the importance of this research.
To assess the correlation between the quality and the categories, we will train a machine learning model, in addition to applying standard mathematical correlation analyses.

\subsection{Stage Two}
In this stage, although we have categorized the commit changes, further work distinguishing between different categories is required.
This is because high-level categories overlap.
In this stage, we will remove the ambiguity of the categories by analyzing the code changes within the commits rather than the commit messages and manual categorizing.
Once this is done, we will investigate the correlation between the categories and changes in code to reveal whether they correlate and how those changes impact software quality.

\subsection{Stage Three}
In the final stage, we will apply our contributions in different ways.
We will construct guidelines which will help developers develop software as well as automate commit message generation and automate classification to support development and maintenance.
In addition, we will create an index which explains how different code patterns impact the quality.
We will conclude this research by releasing them. 

\subsection{Feasibility}
This project is feasible because the requisite data, tools and techniques are readily available:
\begin{itemize}
    \item Data: Open-source software and Git provide sufficient meta-data from Google, Netflix and Apache projects.
    \item Tool: PMD, SonarQube, FindBugs and CAST provide various quality metrics.
    \item Techniques: Machine learning and natural language methods.
\end{itemize}
With all above, we believe this plan will succeed in three to five years.
The midterm milestone is the reasonably high prediction accuracy from the machine learning model, which indicate the categorization is of high quality.
The final milestone is the releases of the new, systematic coding standard and development guidelines and automation tools.
%This direction could succeed since the current techniques in machine learning, statistics, natural language processing will be sufficient to support this research. 

\subsection{Benefits}

We will be able to provide guidelines on how open-source software developers, when contributing to projects,  can improve quality. 
In addition, the results of the second stage will allow us to provide more reliable coding standards and will improve overall code quality.
Improved quality will help to reduce cost and improve software service quality.

\subsection{Intellectual Advancement}

The first goal of this research is to make concrete improvements in the quality of open-software by providing guidelines for developers and, thus, to improve code quality. 
We believe this will also change the way people think, code and develop software.
