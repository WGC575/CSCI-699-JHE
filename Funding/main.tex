% !TeX root = ./main.tex
\documentclass[10pt,conference]{IEEEtran}
\IEEEoverridecommandlockouts
% The preceding line is only needed to identify funding in the first footnote. If that is unneeded, please comment it out.
%\usepackage{cite}
\usepackage{amsmath,amssymb,amsfonts}
\usepackage{algorithmic}
\usepackage{graphicx}
\DeclareGraphicsExtensions{.pdf,.png,.eps}
\usepackage{textcomp}
\usepackage{xcolor}
\usepackage[T1]{fontenc}

\usepackage{xspace}
%new packages from Pooyan's dissertation start
\usepackage{multicol,lipsum}
\usepackage{multirow}
%new packages from Pooyan's dissertation end
\def\BibTeX{{\rm B\kern-.05em{\sc i\kern-.025em b}\kern-.08em
    T\kern-.1667em\lower.7ex\hbox{E}\kern-.125emX}}
    

\begin{document}

\title{Grant Proposal}

\author{
\IEEEauthorblockN{Jincheng He}
\IEEEauthorblockA{\textit{Department of Computer Science} \\
\textit{University of Southern California}\\
%City, Country \\
jinchenh@usc.edu}
}

\maketitle

%\begin{IEEEkeywords}
%Software Engineering, Software Maintenance, Software Quality, Open Source Software
%\end{IEEEkeywords}

\section{Introduction}

%Open Source Software is software that keeps the source code public and the developers often contribute remotely.
%It is a common way to develop software and even large enterprises, such as Google and Netflix  their software 
Developing software with the source code open to the public is very common; however, similar to its closed counterpart, open-source has quality problems, which cause functional failures, such as unsatisfying user experience, and non-functional, such as long responding time.
To improve the quality of open source software, we investigate how it is impacted by commits of different purposes.
By identifying these impacts, we will establish a new set of guidelines for commiting changes, thus improving the quality.

\section{Previous work}

Previous researchers have revealed when, where, how and what the developers contribute to projects and how these aspects impact software quality. 
However, there has been little work on how different categories of commits impact software quality.

\section{Our Plan}

\subsection{Stage One}
Few researchers have studied the correlation between the change type and the code, or how change type impacts quality. 
In this stage, we start by refining the existing categorization of commit changes in open source software repositories.
We evaluate the quality of those changes by obtaining quality metrics from static analysis tools.
To assess the correlation between the quality and the categories, we will train a machine learning model, in addition to applying standard mathematical correlation analyses.

\subsection{Stage Two}
In this stage, although we have categorized the commit changes, further work distinguishing between different categories is required.
This is because high-level categories overlap.
In this stage, we will remove the ambiguity of the categories by analyzing the code changes within the commits rather than the commit messages and manual categorizing.
Once this is done, we will investigate the correlation between the categories and changes in code to reveal whether they correlate and how those changes impact software quality.

\subsection{Stage Three}
In the final stage, we will construct guidelines which will help developers develop software.
In addition, we will create an index which explains how different code patterns impact the quality.
We will conclude this research by completing and validating these two aspects. 

\section{Feasibility}
This project is feasible because the requisite data, tools and techniques are readily available:
\begin{itemize}
    \item Data: Open-source software and Git provide sufficient meta-data from Google, Netflix and Apache projects.
    \item Tool: PMD, SonarQube, FindBugs and CAST provide various quality metrics.
    \item Techniques: Machine learning and natural language methods.
\end{itemize}
With all above, we believe this plan will succeed in three to five years.
The midterm milestone is the reasonably high prediction accuracy from the machine learning model.
The final milestone is the completion of the new systematic coding standard and development guidelines.
%This direction could succeed since the current techniques in machine learning, statistics, natural language processing will be sufficient to support this research. 

\section{Benefits}

We will be able to provide guidelines on how open-source software developers, when contributing to projects,  can improve quality. 
In addition, the results of the second stage will allow us to provide more reliable coding standards and will improve overall code quality.
Improved quality will help to reduce cost and improve software service quality.

\section{Intellectual Advancement}

The first goal of this research is to make concrete improvements in the quality of open-software by providing guidelines for developers and, thus, to improve code quality. 
We believe this will also change the way people think, code and develop software.

\medskip
%\bibliographystyle{./bibliography/IEEEtran}
%\bibliography{./bibliography/IEEEabrv,./bibliography/IEEEexample,./bibliography/references}

\vspace{12pt}
\color{red}
%IEEE conference templates contain guidance text for composing and formatting conference papers. Please ensure that all template text is removed from your conference paper prior to submission to the conference. Failure to remove the template text from your paper may result in your paper not being published.


\end{document}