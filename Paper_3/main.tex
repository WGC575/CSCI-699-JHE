% !TeX root = ./main.tex
\documentclass[10pt,conference]{IEEEtran}
\IEEEoverridecommandlockouts
% The preceding line is only needed to identify funding in the first footnote. If that is unneeded, please comment it out.
\usepackage{cite}
\usepackage{amsmath,amssymb,amsfonts}
\usepackage{algorithmic}
\usepackage{graphicx}
\DeclareGraphicsExtensions{.pdf,.png,.eps}
\usepackage{textcomp}
\usepackage{xcolor}
\usepackage[T1]{fontenc}

\usepackage{xspace}
%new packages from Pooyan's dissertation start
\usepackage{multicol,lipsum}
\usepackage{multirow}
%new packages from Pooyan's dissertation end
\def\BibTeX{{\rm B\kern-.05em{\sc i\kern-.025em b}\kern-.08em
    T\kern-.1667em\lower.7ex\hbox{E}\kern-.125emX}}
    
\definecolor{update}{RGB}{0, 127, 0}
\definecolor{delete}{RGB}{200, 0, 0}
\definecolor{Michael}{RGB}{70, 145, 235}
\definecolor{request}{RGB}{255, 165, 0}
\definecolor{sitao}{RGB}{0,130,0}
\newcommand{\pooyan}[1]{{\leavevmode\color{blue}[Pooyan: #1]}\xspace}
\newcommand{\delete}[1]{{\leavevmode\color{delete}[Deleted: #1]}\xspace}
\newcommand{\michael}[1]{{\leavevmode\color{Michael}[Michael: #1]}\xspace}
\newcommand{\sitao}[1]{{\leavevmode\color{sitao}[sitao: #1]}\xspace}
\newcommand{\update}[1]{{\leavevmode\color{update}[Updated: #1]}\xspace}
\newcommand{\request}[1]{{\leavevmode\color{request}[Requested: #1]}\xspace}
\newcommand{\review}[1]{{\leavevmode\color{orange}[Review: #1]}\xspace}
\newcommand{\RQsize}{\emph{What are the differences between small and large commits in terms of their purpose?}\xspace}
\newcommand{\RQtype}{\emph{Is there any correlation between change type and software compilability?}\xspace}
\newcommand{\RQmetrics}{\emph{Do quality metrics change differently when the software is uncompilable?}\xspace}
%new commands from Pooyan's dissertation start
\newcommand{\algsmall}{\algsetup{linenosize=\small}\small}


\newcommand{\mto}{\mathit{mto}\xspace}
\newcommand{\clusters}[1]{{C_{#1}}}
\newcommand{\entities}[1]{{E_{#1}}}

\newcommand{\abs}[1]{{abs(#1)}}
\newcommand{\card}[1]{{\left\vert #1\right\vert}}

\newcommand{\TODO}[1]{{\leavevmode\color{blue}[: #1]}\xspace}

\newcommand{\RQCompilationEfficiency}{\emph{How effective is my approach in reaching the maximum compilation?}\xspace}
%		 and identifying the impact of commits on compilability?}\xspace}
\newcommand{\RQBrokenSequence}{\emph{What are the characteristics of sequences of uncompilable commits?}\xspace}
\newcommand{\RQBrokenPrediction}{\emph{Is it feasible to predict uncompilability based on commit metadata?}\xspace}
\newcommand{\RQBrokenWhyHow}{\emph{Why do developers commit broken code and how to prevent it?}\xspace}

\newcommand{\RQChangeNeutrals}{\emph{How do quality metrics change when the software is compilable?}\xspace}
\newcommand{\RQQualityProbability}{\emph{How important is analyzing every commit from multiple perspectives?}\xspace}
\newcommand{\RQQualityBroken}{\emph{How do quality metrics change when the software is uncompilable?}\xspace}
\newcommand{\RQQualityCompilability}{\emph{Is there any difference between commits in terms changing quality metrics based on their impact on compilability?}\xspace}

\newcommand{\RQQualityEffectiveness}{\emph{How effective is my approach in identifying change in quality metrics?}\xspace}
\newcommand{\RQArchitectureEvolutionCommit}{\emph{How frequently does the architecture of software change over its commit-history?}\xspace}

\newcommand{\ChapterTitleCompilability}{Compilability Over Commit History\xspace}
\newcommand{\ChapterTitleQuality}{Compilability and Its Impact on Software Quality\xspace}
\newcommand{\ChapterTitleDiscussions}{Discussions\xspace}

\newcommand{\impactful}{\emph{impactful}\xspace}
\newcommand{\iparent}{\emph{impact-parent}\xspace}
\newcommand{\ichild}{\emph{impact-child}\xspace}

\newcommand{\orphan}{\emph{orphan}\xspace}
\newcommand{\simple}{\emph{simple}\xspace}
\newcommand{\merge}{\emph{merge}\xspace}

\newcommand{\broken}{\emph{broken}\xspace}
\newcommand{\solid}{\emph{solid}\xspace}
\newcommand{\breaker}{\emph{breaker}\xspace}
\newcommand{\carrier}{\emph{carrier}\xspace}
\newcommand{\fixer}{\emph{fixer}\xspace}
\newcommand{\neutral}{\emph{neutral}\xspace}

\newcommand{\length}{\emph{length}\xspace}
\newcommand{\duration}{\emph{duration}\xspace}
\newcommand{\involvement}{\emph{involvement}\xspace}

\newcommand{\external}{\emph{external}\xspace}
\newcommand{\affiliated}{\emph{affiliated}\xspace}

\newcommand{\target}{\emph{target}\xspace}
%new commands from Pooyan's dissertation ends

\begin{document}

\title{To be Defined}

\author{
\IEEEauthorblockN{Anonymous}
\IEEEauthorblockA{\textit{Department of Computer Science} \\
\textit{University of Southern California}\\
%City, Country \\
Anonymous@usc.edu}
\and
\IEEEauthorblockN{Anonymous}
\IEEEauthorblockA{\textit{Department of Computer Science} \\
\textit{University of Southern California}\\
%City, Country \\
Anonymous@usc.edu}
}


\maketitle

\begin{abstract}
Developing software with the source code open to the public is very common; however, similar to its closed counterpart, open-source has quality problems, which cause functional failures, such as unsatisfying user experience, and non-functional, such as long responding time.
Previous researchers have revealed when, where, how and what the developers contribute to projects and how these aspects impact software quality. 
However, there has been little work on how different categories of commits impact software quality.
To improve the quality of open source software, we propose this research agenda to investigate how it is impacted by commits of different purposes.
By identifying these impacts, we will establish a new set of guidelines for commiting changes, thus improving the quality.

\end{abstract}

\begin{IEEEkeywords}
Software Engineering, Software Maintenance, Software Quality, Open Source Software
\end{IEEEkeywords}

\section{Introduction}

\subsection{TBD}

The length of commit messages (should be manually written?). Longer commit messages may imply higher software quality.
This research will focus on how different kinds of commits (represented by tags) impact the software quality metrics, including the commit messages and other metrics like security and vulnerability metrics.

\subsection{TBD}

What are the commit messages? We consider commit messages to be another kind of software quality metric. At the same time, it is used to convey the information of contributors of the commented revisions about what they did to the revision to others contributing to the projects, perhaps to other branches.



\section{Related Works}
\cite{Hindle_auto}

%\section{Data Collection}


\section{Research Questions}

\subsection{Do different type of changes have different level of impact on software quality?}

\subsection{Do longer commit messages imply higher software quality?}

\subsection{Is there any relation between commit message lengths and commit types?}

Are those relations causal relations or correlations?

\section{Data Collection}


\section{Our Approach}

\section{Results}
To be defined







\section{Threats to Validity}
TBD

\section{Conclusions}
To be defined

\section*{Acknowledgment}
TBD
%\section*{Acknowledgment}
%This material is based upon work supported in part by the X through the Y under Contract Z.
%X is a Y managed by Z.


\medskip
\bibliographystyle{./bibliography/IEEEtran}
\bibliography{./bibliography/IEEEabrv,./bibliography/IEEEexample,./bibliography/references}

\vspace{12pt}
\color{red}
%IEEE conference templates contain guidance text for composing and formatting conference papers. Please ensure that all template text is removed from your conference paper prior to submission to the conference. Failure to remove the template text from your paper may result in your paper not being published.


\end{document}